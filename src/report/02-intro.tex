\chapter*{Введение}
\addcontentsline{toc}{chapter}{Введение}

На сегодняшний день задача поисковой оптимизации является важнейшей задачей во многих областях. Например, картографические сервисы используют алгоритмы нахождения кратчайшего пути на графе  для нахождения путей между физическими объектами, сервисы перевозки используют алгоритмы нахождения минимальной стоимости пути для сокращения расходов на дорожные затраты. Все эти сервисы стремятся выполнить свою задачу и сохранить как можно больше временных, денежных и человеческих ресурсов. Так возникают задачи оптимизации поиска кратчайшего пути.

Для  решения  задач поисковой оптимизации применяют так называемые популяционные алгоритмы (эволюционные алгоритмы; алгоритмы, использующие  концепцию  роевого  интеллекта;  алгоритмы,  основанные  на иных механизмах живой  и неживой  природы). Наиболее популярными и эффективными являются популяционные алгоритмы, вдохновленные живой природой \cite{karpenko} \cite{eremeev}.

\textbf{Цель данной научно-исследовательской работы} --- провести обзор популяционных алгоритмов, вдохновленных живой природой, для задачи поисковой оптимизации.

Для достижения поставленной цели необходимо решить следующие задачи:
\begin{itemize}
	\item рассмотреть возможные способы решения задачи поисковой оптимизации;
	\item классифицировать популяционные алгоритмы, вдохновленные живой природой;
	\item сравнить описанные алгоритмы по предложенным критериям;
	\item отразить результаты сравнения рассмотренных алгоритмов в выводе.
\end{itemize}


