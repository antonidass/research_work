\chapter*{Заключение}
\addcontentsline{toc}{chapter}{Заключение}

В ходе выполнения данной работы были рассмотрены популяционные алгоритмы, вдохновленные живой природой.

Среди популяционных алгоритмов, вдохновленных живой природой, было уделено внимание муравьиному и пчелиному алгоритмам, а также роевому алгоритму. Роевой алгоритм имеет наибольшую вероятность локализации глобального экстремума, однако низкая скорость сходимости делает его менее привлекательным среди пчелиного и муравьиного алгоритмов. Самая низкая вероятность локализации глобального экстремума оказалась у муравьиного алгоритма, однако данный алгоритм обладает высокой сходимостью.

На основе проделанной работы можно сделать вывод о том, что наиболее подходящим популяционным алгоритмом решения задачи поисковой оптимизации является алгоритм пчелиной колонии, так как данный алгоритм имеет высокую скорость сходимости, а также может быть реализован с помощью средств параллельного программирования, что делает его еще более эффективным. 
 